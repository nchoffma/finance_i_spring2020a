\documentclass[11pt]{article}
\usepackage[margin = 1in]{geometry}
\usepackage{amsmath}
\usepackage{amssymb}
\usepackage{amsthm} % for proof environment
\usepackage{enumitem}
\usepackage{graphicx}
\usepackage{indentfirst}
\usepackage{caption}
\usepackage{lscape}
\usepackage{multirow}
\usepackage{array}

\renewcommand{\labelenumii}{\alph*)}
\newcommand{\w}{\omega}
\newcommand{\p}{\prime}
\newcommand{\one}{\mathbf{1}}
\newcommand{\onep}{\mathbf{1}^\prime}
\newcommand{\lagr}{\mathcal{L}}
\newcommand{\inv}[1]{#1^{-1}}
\newcommand{\ev}{\mathbb{E}}
\renewcommand{\wp}{\omega^\prime}

\begin{document}
\begin{flushleft}
	Nick Hoffman \\
	Finance I, Spring 2020 A \\
	Assignment 1 \\
\end{flushleft}

\begin{enumerate}
	\item Useful facts
	\begin{enumerate}
		\item $ \ev[\w ^\p R ] = \w^\p \bar{R} $, where $\bar{R}$ is the $ N\times 1 $ vector of expected returns.
		\[\ev[\w^\p \bar{R}] = \ev\left[\sum_{i = 1}^{N}\tilde{R}_n \w_n\right] = \sum_{i = 1}^N \w_n \ev[\tilde{R}_n] = \w^\p \bar{R} \]
		
		\item $ \frac{\partial \ev[R^\p \w]}{\partial \w_n} = R_n $, or the entire vector $  \frac{\partial \ev[R^\p \w]}{\partial \w} = R $
		
		As in (a),
		\[\ev[\w^\p \bar{R}] =\sum_{i = 1}^N \w_n \ev[\tilde{R}_n]\]
		and thus
		\[\frac{\partial \ev[R^\p \w]}{\partial \w_n} = \ev[\tilde{R}_n] = \bar{R}\]
		and so in vector form,
		\[ \frac{\partial \ev[R^\p \w]}{\partial \w} = R\]
		
		\item $ V(R^\p \w) = cov(R^\p \w, R^\p \w) $ where $ V $ is the variance-covariance matrix of returns, with typical element $ \sigma_{i,j} = cov(\tilde{R}_i, \tilde{R}_j) $.
		
		By definition, 
		\begin{align*}
		cov(R^\p \w, R^\p \w) &= \ev\left[\sum_{i = 1}^N \w_n \tilde{R}_n \cdot \sum_{i = 1}^N \w_n \tilde{R}_n\right] - \ev\left[\sum_{i = 1}^N \w_n \tilde{R}_n\right]^2 \\
		&= \sum_{i = 1}^N \sum_{j = 1}^N \w_i \w_j \ev[R_i R_j] - \sum_{i = 1}^N \sum_{j = 1}^N \w_i \w_j \ev[R_i] \ev[R_j] \\
		&= \sum_{i = 1}^N \sum_{j = 1}^N \w_i\w_j cov(R_i, R_j) \\
		&= \w^\p V\w
		\end{align*}
		
		\item $ cov(R^\p \w, R_n) = \w^\p V e_n $, where $ e_n = [0, \dots, 0, 1, 0, \dots, 0]^\p $, with 1 in the $ n^{th} $ position.
		
		Using the same definition for the covariance as above, 
		\begin{align*}
		cov(R^\p \w, R_n) &= \ev[R^\p \w R_n] - \ev[R^\p \w]\ev[R_n] \\
		&= \sum_{i = 1}^N \w_i \big(\ev[R_i R_n] - \ev[R_i]\ev[R_n]\big) \\
		&= \sum_{i = 1}^N \w_i cov(R_i, R_n)
		\end{align*} 
		The final sum above is the $ n^{th} $ entry of the matrix $ \w^\p V $, and thus $ cov(R^\p \w, R_n) = \w^\p V e_n $
		
		\item $ \frac{\partial V(R^\p \w)}{\partial \w_n} = 2 cov(R^\p \w, R_n) $
		
		As in part (c), 
		\[V(R^\p \w) = \sum_{i = 1}^N \sum_{j = 1}^N \w_i \w_j cov(R_i, R_j)\]
		Thus,

		\[\frac{\partial V(R^\p \w)}{\partial \w_n} = 2 \sum_{i = 1}^N \w_i cov(R_i, R_n) = 2 cov(R^\p \w, R_n) \]
		where the final equality follows from (d).
	\end{enumerate}
	\item The portfolio problem with risky assets is 
	\begin{gather*}
	\min_\w \frac{1}{2}\w^\p V \w \\
	\text{s.t.} \\
	\w^\p \bar{R} \geq \mu \\
	\w^\p \one = 1
	\end{gather*}
	where $ \one $ is the $N$-vector with 1 in every element. The Lagrangean for this problem is
	\[\lagr = \frac{1}{2}\wp V \w + \lambda_1\big[\mu - \wp\bar{R}\big] + \lambda_2\big[1 - \wp \one\big]\]
	
	with first-order conditions
	\begin{align}
	V\w &= \lambda_1\bar{R} - \lambda_2\one \label{foc_1_1} \\
	\mu &= \wp \bar{R} \label{foc_1_2} \\
	\wp \one &= 1 \label{foc_1_3}
	\end{align}
	To begin, premultiply (\ref{foc_1_1}) by $ \inv{V} $, yielding
	\[\w = \lambda_1 \inv{V}\bar{R} + \lambda_2\inv{V}\one\]
	Now, define the following two efficient portfolios:
	\begin{align*}
	\w_R &= \frac{\inv{V}\bar{R}}{\one^\p \inv{V} \bar{R}} \\
	\w_1 &= \frac{\inv{V} \one}{\one^\p \inv{V} \one}
	\end{align*}
	Thus, the form for $ \w $ above can be transformed in the following way:
	\begin{align*}
	\w^* &= \lambda_1 \inv{V}\bar{R} + \lambda_2\inv{V}\one \\ 
	&= \lambda_1\left(\frac{\one^\p \inv{V}\bar{R}}{\one^\p \inv{V}\bar{R}}\right)\inv{V}\bar{R} + \lambda_2\left(\frac{\one^\p \inv{V}\one}{\one^\p \inv{V}\one}\right)\inv{V}\one \\
	&= \big(\lambda_1\one^\p \inv{V}\bar{R}\big)\w_R + \big(\lambda_2\one^\p \inv{V}\one\big)\w_1 \\
	&\equiv \alpha \w_R + (1 - \alpha)\w_1
	\end{align*}
	
	To show the final equality (i.e., that the weights on portfolios $ \w_R $ and $ w_1 $ sum to unity), we premultiply both sides by $ \one^\p $ and impose  (\ref{foc_1_3}):
	\[\onep \w^*= \big(\lambda_1\one^\p \inv{V}\bar{R}\big)\onep w_R + \big(\lambda_2\one^\p \inv{V}\one\big)\onep w_1 \]
	Because $ \w^* $, $ \w_R $, and $ \w_1 $ are portfolios, the above implies that 
	\[1 = \big(\lambda_1\one^\p \inv{V}\bar{R}\big) + \big(\lambda_2\one^\p \inv{V}\one\big)\]
	To solve for $\alpha$, we impose (\ref{foc_1_2}):
	\begin{align*}
	\w^{*\p}\bar{R} = \mu &\implies \alpha \w_R^\p \bar{R} + (1 - \alpha) \w_1^\p \bar{R} = \mu \\
	& \implies \alpha = \frac{\mu - \w_1^\p \bar{R}}{\w_R^\p \bar{R} - \w_1^\p \bar{R}}
	\end{align*}
	
	\item To find the Global Minimum Variance (GMV) portfolio, the problem is similar:
	\begin{gather*}
	\min_\w \frac{1}{2}\w^\p V \w \\
	\text{s.t.} \\
	\w^\p \one = 1
	\end{gather*}
	
	The Lagrangean for this problem is 
	\[\lagr = \frac{1}{2} \wp V \w + \lambda[1 - \wp \one ]\]
	From the first-order condition for $\omega$, we can use the same method as in question 2:
	\begin{align*}
	\w^* &= \lambda \inv{V} \one  \\
	&= \lambda\left(\frac{\one^\p \inv{V}\one}{\one^\p \inv{V}\one}\right)\inv{V}\one \\
	&= \big(\lambda \onep \inv{V}\one \big)\w_1
	\end{align*}
	Because $ \w $ and $ \w_1 $ are both portfolios, premultiplying the final line above by $ \onep $ shows that $ (\lambda \onep \inv{V}\one) = 1 $. Thus, $ \w_1 $ is the GMV portfolio
	
	\item With both risky assets and one risk-free asset, the problem becomes
	\begin{gather*}
	\min_\w \frac{1}{2}\w^\p V \w \\
	\text{s.t.} \\
	(1 - \w^\p \one)R_f + \w^\p \bar{R} \geq \mu
	\end{gather*}
	\begin{enumerate}
		\item The Lagrangian for this problem is as follows:
		\[\lagr = \frac{1}{2}\w^\p V \w + \lambda \big[\mu - (1 - \w^\p \one)R_f - \w^\p \bar{R}\big] \]
		with first-order conditions
		\begin{align}
		V\w + \lambda(\one R_f - \bar{R}) &= \mu \label{foc_4_1}\\
		(1 - \w^\p \one)R_f + \w^\p \bar{R} &\geq \mu \label{foc_4_2}
		\end{align}
		To begin, premultiply (\ref{foc_4_1}) by $ \inv{V} $ (assuming that $ V $ is invertible):

		\begin{equation}\label{step2_4}
		\w = \lambda \inv{V} \big(\bar{R} - \one R_f\big)
		\end{equation}
		
		Similarly, from (\ref{foc_4_2}), 
		\[\big(\bar{R} - \one R_f\big)\wp = \mu - R_f\]
		Thus, premultiplying (\ref{step2_4}) by $ (\bar{R} - \one R_f) $ gives
		\[\lambda\big(\bar{R} - \one R_f\big) \inv{V} \big(\bar{R} - \one R_f\big)\]
		and thus
		\[\lambda = \frac{\mu - R_f}{(\bar{R} - \one R_f\big) \inv{V} \big(\bar{R} - \one R_f )}\]
		Therefore, the optimal $ \w $ is given by 
		\[\w = \frac{\mu - R_f}{(\bar{R} - \one R_f\big) \inv{V} \big(\bar{R} - \one R_f )} \inv{V} \big(\bar{R} - \one R_f\big) \]
		
		\item To verify that the efficient frontier is linear in $ \sigma_p $, note that the variance of this portfolio is given by 
		\begin{align*}
		\wp V\w &= \frac{\mu - R_f}{(\bar{R} - \one R_f\big) \inv{V} \big(\bar{R} - \one R_f )} \big(\bar{R} - \one R_f\big)^\p \inv{V} V \frac{\mu - R_f}{(\bar{R} - \one R_f\big) \inv{V} \big(\bar{R} - \one R_f )} \inv{V} \big(\bar{R} - \one R_f\big) \\
		&= \frac{(\mu - R_f)^2}{\inv{V} \big(\bar{R} - \one R_f\big)}
		\end{align*}
		And thus the variance is 
		\[\sigma_p = \sqrt{\wp V \w} = \frac{|\mu - R_f}{\sqrt{\inv{V} \big(\bar{R} - \one R_f\big)}}\]
		Thus, solving for the return $ \mu_p $ gives
		\[\mu_p = R_f \pm \sigma_p \sqrt{\inv{V} \big(\bar{R} - \one R_f\big)}\]
		Thus, the portfolio return is linear in the variance, and thus the efficient frontier is linear. 
		
		\item The optimal Sharpe Ratio portfolio is the portfolio $ \w $ which solves
		\[\max_\w \frac{\wp\bar{R} - R_f}{\sqrt{\wp V \w}}\]
		The first-order condition for this problem is 
		\[\frac{d}{d\w} = -\frac{1}{2}(\wp \bar{R}) - R_f (\wp V \w)^{-\frac{3}{2}} V\w + \bar{R}(\wp V \w)^{-\frac{1}{2}} = 0 \]
		
		The portfolio solving this problem is 
		\[\w^* = \frac{\inv{V}\big(\bar{R} - \one R_f\big)}{\onep \inv{V} \big(\bar{R} - \one R_f\big)} \]
		
		This is the portfolio where, given $ N $ risky assets, the efficient frontier for the assets is tangent to the efficient frontier for the risky assets plus a risk free asset. Thus, this portfolio is also the efficient portfolio that includes only risky assets, when both risky and risk-free assets are available. 
	\end{enumerate}
\end{enumerate}

\end{document}